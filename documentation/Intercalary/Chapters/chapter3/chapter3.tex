\chapter{Solution}

In this chapter we will present the Jayield solution, going through it's different components and understanding how they work and interact with each other.
\section{Yield}

Yield is the interface that represents the element generator in our solution. It's interface is equivalent to a Java Consumer, however, it has a few key differences. Listing~\ref{lst:yield} presents the definition of the Yield interface.

\lstinputlisting[caption={Yield},label={lst:yield}]{./Chapters/chapter3/listings/Yield.java}

As we can see Yield has two methods:
\begin{itemize}
\item bye(), which is used to short circuit the traversal of a sequence.
\item ret(T item), which is the method used to define what to do with each item that a sequence returns when traversing it.
\end{itemize}

Lets say we were traversing a sequence of Fruit names and we would like to write down their names, listing~\ref{lst:yield-example} shows what the Yield lambda to achieve this looks like.

\lstinputlisting[caption={Yield example},label={lst:yield-example}]{./Chapters/chapter3/listings/YieldExample.java}

\section{Traverser}

Traverser is an interface that is used to define ways of bulk travesing a sequence. To that end it provides the method traverse which receives a Yield function as argument to consume elements as they are traversed. Listing~\ref{lst:traverser} shows the definition of the Traverser interface.

\lstinputlisting[caption={Traverser},label={lst:traverser}]{./Chapters/chapter3/listings/Traverser.java}

Looking at some examples on how to use this interface, if we have an upstream sequence of Fruits, listing~\ref{lst:traverser-map-example} shows of how to transform such a sequence into a sequence of Fruit's names.

\lstinputlisting[caption={Traverser map to name example},label={lst:traverser-map-example}]{./Chapters/chapter3/listings/TraverserMapExample.java}

\newpage
We can also define new sequences with the Traverser interface, lets say we want to generate a sequence of Fruits, listing~\ref{lst:traverser-generator-example} shows an example of how to do it.

\lstinputlisting[caption={Traverser as sequence generator},label={lst:traverser-generator-example}]{./Chapters/chapter3/listings/TraverserAsSequenceGenerator.java}

\section{Query}

Looking now at the Query class in Jayield, it represents a sequence of elements supporting sequential operations. This class implements all the utility methods with the same name, semantics and behaviour as those provided by Java's Stream. This class has an Traverser field that represents the current sequence. It also provides a way for the user to define new operations through the then method (listing~\ref{lst:query-then}).

\lstinputlisting[caption={Query then method},label={lst:query-then}]{./Chapters/chapter3/listings/QueryThen.java}

 This method receives a Function as parameter that defines the way to transform the upstream sequence (represented by a Query) into the new traversal function, a Traverser. This function is similar to the method presented in the previous section in listing~\ref{lst:traverser-map-example}, with that in listing~\ref{lst:traverser-map-example} we receive a Traverser as a parameter and the Function parameter in next receives a Query. An example of how to use this method can be looked at back in Chapter 1's listing~\ref{lst:jayield-filterOdd}.

\section{Advancer}

The Advancer is the interface that represents our individual traverser and the main focus of the use case of we are implementing.

The implementation for the Advancer (i.e. the individual traverser) will only be generated if requested to the Advancer interface as it is not always necessary to traverse. To do so we can call one of the two from methods provided in Advancer as shown in the~\ref{lst:advancer} listing.

\lstinputlisting[caption={Advancer},label={lst:advancer}]{./Chapters/chapter3/listings/Advancer.java}

This interface provides a way for the programmer to individually traverse a sequence, either through the tryAdvance method or through the use of an iterator. For instance, if we wanted to print the name of the first Fruit in a sequence we could take the approach shown in listing ~\ref{lst:advancer-example} which will print the first fruit's name, and only the first, if it exists.

\lstinputlisting[caption={Advancer example},label={lst:advancer-example}]{./Chapters/chapter3/listings/AdvancerExample.java}

So, the question now is, how do we generate an Advancer from a Query or from a Traverser?


\subsection{Approach}

To achieve the expected result a pseudo-solution was thought of, assuming one of the following is true:
\begin{itemize}
\item A call to source.traverse(...) is made
\item A call to yield.ret(...) is made
\end{itemize}


Assuming any of these are true, we would generate an Advancer doing the following:
\begin{itemize}
\item Instead of calling source.traverse(...) we would need to call source.tryAdvance(...)
\item When calling yield.ret(...) we would register that an element has been found that meets the criteria.
\item The call to source.tryAdvance(...) is made like so: while(noElementFound \&\& source.tryAdvance(...))
\end{itemize}

So, lets say we have the following definition for a map Traverser:
\begin{lstlisting}[caption={Map Traverser},label={lst:map-traverser},captionpos=b]
<T, R> Traverser<R> map(Query<T> source, Function<T, R> mapper) {
	return yield -> {
    	source.traverse(e -> yield.ret(mapper.apply(e)));
	};
}
\end{lstlisting}

Given the bulk traversal definition of map in listing~\ref{lst:map-traverser}, then the resulting map which implements the individual traversal is shown in listing~\ref{lst:map-advancer}

\begin{lstlisting}[caption={Map Advancer},label={lst:map-advancer},captionpos=b]
<T, R> Advancer<R> map(Query<T> source, Function<T, R> mapper) {
	(*@\textcolor{blue}{boolean[] noElementFound = new boolean[] \{true\};}@*)
	return yield -> {
		(*@\textcolor{blue}{noElementFound[0] = true;}@*)
		while(noElementFound[0] && source.tryAdvance(e -> {
				yield.ret(mapper.apply(e)));
				(*@\textcolor{blue}{noElementFound[0] = false;}@*)
			});
		(*@\textcolor{blue}{return noElementFound[0] == false;}@*)
	};
}
\end{lstlisting}

With that said, there are corner cases that are not so easily mapped as is the case of the flatMap operation which traverses over multiple sequences, these cases need a different approach.

Nevertheless, following the suggested approach we will need to generate code at runtime, to do this we will use ASM, a library for java bytecode manipulation at runtime. In order to read the bytecode via ASM we first need to give it (via the ClassReader) the actual bytecode of the Traverser. The problem is, if the Traverser was generated using a lambda function, that the lambda's class is generated at runtime via the LambdaMetaFactory class and therefore the normal method of getting the bytecode does not work, aside from that we also needed to get any fields the Traverser's class may have, in the case of the lambda, they would be captured arguments.

\subsection{SerializedLambda}

You may have noticed in listing~\ref{lst:traverser} that the Traverser interface extends from Serializable, the fact is, we actually need the Traverser to extend Serializable in order to read its code. This solution was found in an answer to a question in Stackoverflow\citep{stackoverflowlambdacode},which makes use of the SerializedLambda\citep{serializedlambda}.

The reason for this is that, if the lambda we are trying to read implements an interface that extends Serializable, the LambdaMetaFactory class will generate a private method called "writeReplace" which provides an instance of the class SerializedLambda. This instance can be used to retrieve both the actual method that was generated for this lambda as well as get the captured arguments and being able to reach the lambda's method allow us to instrument it.

\subsection{Visitors}

Now having access to the lambda's code, it was time to instrument it in order to transform the Traverser into an Advancer. To do so, we started by using a custom ClassVisitor that copies the original declaring class of the lambda as well as making some changes along the way, such as:
\begin{itemize}
\item The name of the class would now be the name of the lambda's method.
\item Each call to a method on the declared class would now be a call to the generated class, due to calls to private methods.
\item The return type of the lambda's method would now be boolean instead of void, to be coherent with the Advancer Interface.
\item The lambda's method would now call tryAdvance instead of traverse and return the result of tryAdvance as well.
\end{itemize}

To obtain an Advancer out of the generated class all we would need to do now would be:
\begin{itemize}
\item Retrieve the original lambda's captured arguments
\item Retrieve the generated method from the new class
\item Apply the approach defined above.
\end{itemize}

Which would result in something like this:
\begin{lstlisting}[caption={Generated Advancer},captionpos=b, label={lst:advanger-gen1}]
final BoolBox elementFound = new BoolBox();
Method generatedTryAdvance = getTryAdvance(...);
Advancer generated =  y -> {
	elementFound.reset();
	Yield<R> wrapper = wr -> {
		y.ret(wr);
		elementFound.set();
	};
	arguments[arguments.length - 1] = wrapper;
	while(elementFound.isFalse() && (Boolean) generatedTryAdvance.invoke(newClass, arguments));
	return elementFound.isTrue();
};
\end{lstlisting}

This solution makes use of a wrapper over the yield, received as parameter, to register when an element is found, this wrapper is then provided as the last argument in an argument array for the generated method, with the first parameters being those captured by the lambda. 

\subsection{State Machine}

This approach however, as said before, has flaws and the above described solution does not work for every case. One of the first issues identified was the case functions that would generate more elements than the original sequence, that being either, a duplicate function or a generator of values. Lets look at listing~\ref{lst:traverser-dup} as an example of an traverser that duplicates values.

\lstinputlisting[caption={Duplicate Traverser function},label={lst:traverser-dup}]{./Chapters/chapter3/listings/TraverserDup.java}

For this Traverser, the solution described above would not work as it would not stop when the first element was found nor would it return to the same point when the next element was requested. To solve this issue two things were necessary, a state machine and external iteration.

The first approach to this issue was based on declaring the end of a state, and the beginning of the next, every time a yield.ret(...) was made. This would translate to something like what is shown in listing~\ref{lst:advancer-dup-ifs}.

\lstinputlisting[caption={Duplicate Advancer translation with if State Machine},label={lst:advancer-dup-ifs}]{./Chapters/chapter3/listings/AdvancerDupIfs.java}

As you can see there are a few key differences:
\begin{itemize}
\item the instrumented yield operation now isolates yield.ret() calls into different states
\item the validValue field, which lets us know if we need to get a new value or if the old one still needs to be used in another state.
\end{itemize}

This solution had it's problems aswell, for instance, if a yield return was made inside an if condition it would have to take that into account which would lead to even more complicated translations.

At this moment we took a step back to re-evaluate and it was then that we started to look into Jon Skeet's C\# in Depth\citep{csharpindepth}, chapter 6. The approach described here, makes use of a switch case as well as labels, with the labels being the delimiters between the end of a state and the beginning of a new one. This approach requires some pre-processing of the Traverser code, to know how many states there will be in the switch as in Java to declare a switch we need to know in advance how many states there will be, this approach was also applied to value generating lambdas, as the same principles apply. Listing~\ref{lst:advancer-dup-switch} shows an example of the translation of the dup operation.

\lstinputlisting[caption={Duplicate Advancer translation with Switch based State Machine},label={lst:advancer-dup-switch}]{./Chapters/chapter3/listings/AdvancerDupSwitch.java}